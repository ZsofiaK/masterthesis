\documentclass[11pt]{article}

    \usepackage[breakable]{tcolorbox}
    \usepackage{parskip} % Stop auto-indenting (to mimic markdown behaviour)
    

    % Basic figure setup, for now with no caption control since it's done
    % automatically by Pandoc (which extracts ![](path) syntax from Markdown).
    \usepackage{graphicx}
    % Maintain compatibility with old templates. Remove in nbconvert 6.0
    \let\Oldincludegraphics\includegraphics
    % Ensure that by default, figures have no caption (until we provide a
    % proper Figure object with a Caption API and a way to capture that
    % in the conversion process - todo).
    \usepackage{caption}
    \DeclareCaptionFormat{nocaption}{}
    \captionsetup{format=nocaption,aboveskip=0pt,belowskip=0pt}

    \usepackage{float}
    \floatplacement{figure}{H} % forces figures to be placed at the correct location
    \usepackage{xcolor} % Allow colors to be defined
    \usepackage{enumerate} % Needed for markdown enumerations to work
    \usepackage{geometry} % Used to adjust the document margins
    \usepackage{amsmath} % Equations
    \usepackage{amssymb} % Equations
    \usepackage{textcomp} % defines textquotesingle
    % Hack from http://tex.stackexchange.com/a/47451/13684:
    \AtBeginDocument{%
        \def\PYZsq{\textquotesingle}% Upright quotes in Pygmentized code
    }
    \usepackage{upquote} % Upright quotes for verbatim code
    \usepackage{eurosym} % defines \euro

    \usepackage{iftex}
    \ifPDFTeX
        \usepackage[T1]{fontenc}
        \IfFileExists{alphabeta.sty}{
              \usepackage{alphabeta}
          }{
              \usepackage[mathletters]{ucs}
              \usepackage[utf8x]{inputenc}
          }
    \else
        \usepackage{fontspec}
        \usepackage{unicode-math}
    \fi

    \usepackage{fancyvrb} % verbatim replacement that allows latex
    \usepackage{grffile} % extends the file name processing of package graphics
                         % to support a larger range
    \makeatletter % fix for old versions of grffile with XeLaTeX
    \@ifpackagelater{grffile}{2019/11/01}
    {
      % Do nothing on new versions
    }
    {
      \def\Gread@@xetex#1{%
        \IfFileExists{"\Gin@base".bb}%
        {\Gread@eps{\Gin@base.bb}}%
        {\Gread@@xetex@aux#1}%
      }
    }
    \makeatother
    \usepackage[Export]{adjustbox} % Used to constrain images to a maximum size
    \adjustboxset{max size={0.9\linewidth}{0.9\paperheight}}

    % The hyperref package gives us a pdf with properly built
    % internal navigation ('pdf bookmarks' for the table of contents,
    % internal cross-reference links, web links for URLs, etc.)
    \usepackage{hyperref}
    % The default LaTeX title has an obnoxious amount of whitespace. By default,
    % titling removes some of it. It also provides customization options.
    \usepackage{titling}
    \usepackage{longtable} % longtable support required by pandoc >1.10
    \usepackage{booktabs}  % table support for pandoc > 1.12.2
    \usepackage{array}     % table support for pandoc >= 2.11.3
    \usepackage{calc}      % table minipage width calculation for pandoc >= 2.11.1
    \usepackage[inline]{enumitem} % IRkernel/repr support (it uses the enumerate* environment)
    \usepackage[normalem]{ulem} % ulem is needed to support strikethroughs (\sout)
                                % normalem makes italics be italics, not underlines
    \usepackage{mathrsfs}
    

    
    % Colors for the hyperref package
    \definecolor{urlcolor}{rgb}{0,.145,.698}
    \definecolor{linkcolor}{rgb}{.71,0.21,0.01}
    \definecolor{citecolor}{rgb}{.12,.54,.11}

    % ANSI colors
    \definecolor{ansi-black}{HTML}{3E424D}
    \definecolor{ansi-black-intense}{HTML}{282C36}
    \definecolor{ansi-red}{HTML}{E75C58}
    \definecolor{ansi-red-intense}{HTML}{B22B31}
    \definecolor{ansi-green}{HTML}{00A250}
    \definecolor{ansi-green-intense}{HTML}{007427}
    \definecolor{ansi-yellow}{HTML}{DDB62B}
    \definecolor{ansi-yellow-intense}{HTML}{B27D12}
    \definecolor{ansi-blue}{HTML}{208FFB}
    \definecolor{ansi-blue-intense}{HTML}{0065CA}
    \definecolor{ansi-magenta}{HTML}{D160C4}
    \definecolor{ansi-magenta-intense}{HTML}{A03196}
    \definecolor{ansi-cyan}{HTML}{60C6C8}
    \definecolor{ansi-cyan-intense}{HTML}{258F8F}
    \definecolor{ansi-white}{HTML}{C5C1B4}
    \definecolor{ansi-white-intense}{HTML}{A1A6B2}
    \definecolor{ansi-default-inverse-fg}{HTML}{FFFFFF}
    \definecolor{ansi-default-inverse-bg}{HTML}{000000}

    % common color for the border for error outputs.
    \definecolor{outerrorbackground}{HTML}{FFDFDF}

    % commands and environments needed by pandoc snippets
    % extracted from the output of `pandoc -s`
    \providecommand{\tightlist}{%
      \setlength{\itemsep}{0pt}\setlength{\parskip}{0pt}}
    \DefineVerbatimEnvironment{Highlighting}{Verbatim}{commandchars=\\\{\}}
    % Add ',fontsize=\small' for more characters per line
    \newenvironment{Shaded}{}{}
    \newcommand{\KeywordTok}[1]{\textcolor[rgb]{0.00,0.44,0.13}{\textbf{{#1}}}}
    \newcommand{\DataTypeTok}[1]{\textcolor[rgb]{0.56,0.13,0.00}{{#1}}}
    \newcommand{\DecValTok}[1]{\textcolor[rgb]{0.25,0.63,0.44}{{#1}}}
    \newcommand{\BaseNTok}[1]{\textcolor[rgb]{0.25,0.63,0.44}{{#1}}}
    \newcommand{\FloatTok}[1]{\textcolor[rgb]{0.25,0.63,0.44}{{#1}}}
    \newcommand{\CharTok}[1]{\textcolor[rgb]{0.25,0.44,0.63}{{#1}}}
    \newcommand{\StringTok}[1]{\textcolor[rgb]{0.25,0.44,0.63}{{#1}}}
    \newcommand{\CommentTok}[1]{\textcolor[rgb]{0.38,0.63,0.69}{\textit{{#1}}}}
    \newcommand{\OtherTok}[1]{\textcolor[rgb]{0.00,0.44,0.13}{{#1}}}
    \newcommand{\AlertTok}[1]{\textcolor[rgb]{1.00,0.00,0.00}{\textbf{{#1}}}}
    \newcommand{\FunctionTok}[1]{\textcolor[rgb]{0.02,0.16,0.49}{{#1}}}
    \newcommand{\RegionMarkerTok}[1]{{#1}}
    \newcommand{\ErrorTok}[1]{\textcolor[rgb]{1.00,0.00,0.00}{\textbf{{#1}}}}
    \newcommand{\NormalTok}[1]{{#1}}

    % Additional commands for more recent versions of Pandoc
    \newcommand{\ConstantTok}[1]{\textcolor[rgb]{0.53,0.00,0.00}{{#1}}}
    \newcommand{\SpecialCharTok}[1]{\textcolor[rgb]{0.25,0.44,0.63}{{#1}}}
    \newcommand{\VerbatimStringTok}[1]{\textcolor[rgb]{0.25,0.44,0.63}{{#1}}}
    \newcommand{\SpecialStringTok}[1]{\textcolor[rgb]{0.73,0.40,0.53}{{#1}}}
    \newcommand{\ImportTok}[1]{{#1}}
    \newcommand{\DocumentationTok}[1]{\textcolor[rgb]{0.73,0.13,0.13}{\textit{{#1}}}}
    \newcommand{\AnnotationTok}[1]{\textcolor[rgb]{0.38,0.63,0.69}{\textbf{\textit{{#1}}}}}
    \newcommand{\CommentVarTok}[1]{\textcolor[rgb]{0.38,0.63,0.69}{\textbf{\textit{{#1}}}}}
    \newcommand{\VariableTok}[1]{\textcolor[rgb]{0.10,0.09,0.49}{{#1}}}
    \newcommand{\ControlFlowTok}[1]{\textcolor[rgb]{0.00,0.44,0.13}{\textbf{{#1}}}}
    \newcommand{\OperatorTok}[1]{\textcolor[rgb]{0.40,0.40,0.40}{{#1}}}
    \newcommand{\BuiltInTok}[1]{{#1}}
    \newcommand{\ExtensionTok}[1]{{#1}}
    \newcommand{\PreprocessorTok}[1]{\textcolor[rgb]{0.74,0.48,0.00}{{#1}}}
    \newcommand{\AttributeTok}[1]{\textcolor[rgb]{0.49,0.56,0.16}{{#1}}}
    \newcommand{\InformationTok}[1]{\textcolor[rgb]{0.38,0.63,0.69}{\textbf{\textit{{#1}}}}}
    \newcommand{\WarningTok}[1]{\textcolor[rgb]{0.38,0.63,0.69}{\textbf{\textit{{#1}}}}}


    % Define a nice break command that doesn't care if a line doesn't already
    % exist.
    \def\br{\hspace*{\fill} \\* }
    % Math Jax compatibility definitions
    \def\gt{>}
    \def\lt{<}
    \let\Oldtex\TeX
    \let\Oldlatex\LaTeX
    \renewcommand{\TeX}{\textrm{\Oldtex}}
    \renewcommand{\LaTeX}{\textrm{\Oldlatex}}
    % Document parameters
    % Document title
    \title{ZKatona\_thesis\_EDA}
    
    
    
    
    
    
    
% Pygments definitions
\makeatletter
\def\PY@reset{\let\PY@it=\relax \let\PY@bf=\relax%
    \let\PY@ul=\relax \let\PY@tc=\relax%
    \let\PY@bc=\relax \let\PY@ff=\relax}
\def\PY@tok#1{\csname PY@tok@#1\endcsname}
\def\PY@toks#1+{\ifx\relax#1\empty\else%
    \PY@tok{#1}\expandafter\PY@toks\fi}
\def\PY@do#1{\PY@bc{\PY@tc{\PY@ul{%
    \PY@it{\PY@bf{\PY@ff{#1}}}}}}}
\def\PY#1#2{\PY@reset\PY@toks#1+\relax+\PY@do{#2}}

\@namedef{PY@tok@w}{\def\PY@tc##1{\textcolor[rgb]{0.73,0.73,0.73}{##1}}}
\@namedef{PY@tok@c}{\let\PY@it=\textit\def\PY@tc##1{\textcolor[rgb]{0.24,0.48,0.48}{##1}}}
\@namedef{PY@tok@cp}{\def\PY@tc##1{\textcolor[rgb]{0.61,0.40,0.00}{##1}}}
\@namedef{PY@tok@k}{\let\PY@bf=\textbf\def\PY@tc##1{\textcolor[rgb]{0.00,0.50,0.00}{##1}}}
\@namedef{PY@tok@kp}{\def\PY@tc##1{\textcolor[rgb]{0.00,0.50,0.00}{##1}}}
\@namedef{PY@tok@kt}{\def\PY@tc##1{\textcolor[rgb]{0.69,0.00,0.25}{##1}}}
\@namedef{PY@tok@o}{\def\PY@tc##1{\textcolor[rgb]{0.40,0.40,0.40}{##1}}}
\@namedef{PY@tok@ow}{\let\PY@bf=\textbf\def\PY@tc##1{\textcolor[rgb]{0.67,0.13,1.00}{##1}}}
\@namedef{PY@tok@nb}{\def\PY@tc##1{\textcolor[rgb]{0.00,0.50,0.00}{##1}}}
\@namedef{PY@tok@nf}{\def\PY@tc##1{\textcolor[rgb]{0.00,0.00,1.00}{##1}}}
\@namedef{PY@tok@nc}{\let\PY@bf=\textbf\def\PY@tc##1{\textcolor[rgb]{0.00,0.00,1.00}{##1}}}
\@namedef{PY@tok@nn}{\let\PY@bf=\textbf\def\PY@tc##1{\textcolor[rgb]{0.00,0.00,1.00}{##1}}}
\@namedef{PY@tok@ne}{\let\PY@bf=\textbf\def\PY@tc##1{\textcolor[rgb]{0.80,0.25,0.22}{##1}}}
\@namedef{PY@tok@nv}{\def\PY@tc##1{\textcolor[rgb]{0.10,0.09,0.49}{##1}}}
\@namedef{PY@tok@no}{\def\PY@tc##1{\textcolor[rgb]{0.53,0.00,0.00}{##1}}}
\@namedef{PY@tok@nl}{\def\PY@tc##1{\textcolor[rgb]{0.46,0.46,0.00}{##1}}}
\@namedef{PY@tok@ni}{\let\PY@bf=\textbf\def\PY@tc##1{\textcolor[rgb]{0.44,0.44,0.44}{##1}}}
\@namedef{PY@tok@na}{\def\PY@tc##1{\textcolor[rgb]{0.41,0.47,0.13}{##1}}}
\@namedef{PY@tok@nt}{\let\PY@bf=\textbf\def\PY@tc##1{\textcolor[rgb]{0.00,0.50,0.00}{##1}}}
\@namedef{PY@tok@nd}{\def\PY@tc##1{\textcolor[rgb]{0.67,0.13,1.00}{##1}}}
\@namedef{PY@tok@s}{\def\PY@tc##1{\textcolor[rgb]{0.73,0.13,0.13}{##1}}}
\@namedef{PY@tok@sd}{\let\PY@it=\textit\def\PY@tc##1{\textcolor[rgb]{0.73,0.13,0.13}{##1}}}
\@namedef{PY@tok@si}{\let\PY@bf=\textbf\def\PY@tc##1{\textcolor[rgb]{0.64,0.35,0.47}{##1}}}
\@namedef{PY@tok@se}{\let\PY@bf=\textbf\def\PY@tc##1{\textcolor[rgb]{0.67,0.36,0.12}{##1}}}
\@namedef{PY@tok@sr}{\def\PY@tc##1{\textcolor[rgb]{0.64,0.35,0.47}{##1}}}
\@namedef{PY@tok@ss}{\def\PY@tc##1{\textcolor[rgb]{0.10,0.09,0.49}{##1}}}
\@namedef{PY@tok@sx}{\def\PY@tc##1{\textcolor[rgb]{0.00,0.50,0.00}{##1}}}
\@namedef{PY@tok@m}{\def\PY@tc##1{\textcolor[rgb]{0.40,0.40,0.40}{##1}}}
\@namedef{PY@tok@gh}{\let\PY@bf=\textbf\def\PY@tc##1{\textcolor[rgb]{0.00,0.00,0.50}{##1}}}
\@namedef{PY@tok@gu}{\let\PY@bf=\textbf\def\PY@tc##1{\textcolor[rgb]{0.50,0.00,0.50}{##1}}}
\@namedef{PY@tok@gd}{\def\PY@tc##1{\textcolor[rgb]{0.63,0.00,0.00}{##1}}}
\@namedef{PY@tok@gi}{\def\PY@tc##1{\textcolor[rgb]{0.00,0.52,0.00}{##1}}}
\@namedef{PY@tok@gr}{\def\PY@tc##1{\textcolor[rgb]{0.89,0.00,0.00}{##1}}}
\@namedef{PY@tok@ge}{\let\PY@it=\textit}
\@namedef{PY@tok@gs}{\let\PY@bf=\textbf}
\@namedef{PY@tok@gp}{\let\PY@bf=\textbf\def\PY@tc##1{\textcolor[rgb]{0.00,0.00,0.50}{##1}}}
\@namedef{PY@tok@go}{\def\PY@tc##1{\textcolor[rgb]{0.44,0.44,0.44}{##1}}}
\@namedef{PY@tok@gt}{\def\PY@tc##1{\textcolor[rgb]{0.00,0.27,0.87}{##1}}}
\@namedef{PY@tok@err}{\def\PY@bc##1{{\setlength{\fboxsep}{\string -\fboxrule}\fcolorbox[rgb]{1.00,0.00,0.00}{1,1,1}{\strut ##1}}}}
\@namedef{PY@tok@kc}{\let\PY@bf=\textbf\def\PY@tc##1{\textcolor[rgb]{0.00,0.50,0.00}{##1}}}
\@namedef{PY@tok@kd}{\let\PY@bf=\textbf\def\PY@tc##1{\textcolor[rgb]{0.00,0.50,0.00}{##1}}}
\@namedef{PY@tok@kn}{\let\PY@bf=\textbf\def\PY@tc##1{\textcolor[rgb]{0.00,0.50,0.00}{##1}}}
\@namedef{PY@tok@kr}{\let\PY@bf=\textbf\def\PY@tc##1{\textcolor[rgb]{0.00,0.50,0.00}{##1}}}
\@namedef{PY@tok@bp}{\def\PY@tc##1{\textcolor[rgb]{0.00,0.50,0.00}{##1}}}
\@namedef{PY@tok@fm}{\def\PY@tc##1{\textcolor[rgb]{0.00,0.00,1.00}{##1}}}
\@namedef{PY@tok@vc}{\def\PY@tc##1{\textcolor[rgb]{0.10,0.09,0.49}{##1}}}
\@namedef{PY@tok@vg}{\def\PY@tc##1{\textcolor[rgb]{0.10,0.09,0.49}{##1}}}
\@namedef{PY@tok@vi}{\def\PY@tc##1{\textcolor[rgb]{0.10,0.09,0.49}{##1}}}
\@namedef{PY@tok@vm}{\def\PY@tc##1{\textcolor[rgb]{0.10,0.09,0.49}{##1}}}
\@namedef{PY@tok@sa}{\def\PY@tc##1{\textcolor[rgb]{0.73,0.13,0.13}{##1}}}
\@namedef{PY@tok@sb}{\def\PY@tc##1{\textcolor[rgb]{0.73,0.13,0.13}{##1}}}
\@namedef{PY@tok@sc}{\def\PY@tc##1{\textcolor[rgb]{0.73,0.13,0.13}{##1}}}
\@namedef{PY@tok@dl}{\def\PY@tc##1{\textcolor[rgb]{0.73,0.13,0.13}{##1}}}
\@namedef{PY@tok@s2}{\def\PY@tc##1{\textcolor[rgb]{0.73,0.13,0.13}{##1}}}
\@namedef{PY@tok@sh}{\def\PY@tc##1{\textcolor[rgb]{0.73,0.13,0.13}{##1}}}
\@namedef{PY@tok@s1}{\def\PY@tc##1{\textcolor[rgb]{0.73,0.13,0.13}{##1}}}
\@namedef{PY@tok@mb}{\def\PY@tc##1{\textcolor[rgb]{0.40,0.40,0.40}{##1}}}
\@namedef{PY@tok@mf}{\def\PY@tc##1{\textcolor[rgb]{0.40,0.40,0.40}{##1}}}
\@namedef{PY@tok@mh}{\def\PY@tc##1{\textcolor[rgb]{0.40,0.40,0.40}{##1}}}
\@namedef{PY@tok@mi}{\def\PY@tc##1{\textcolor[rgb]{0.40,0.40,0.40}{##1}}}
\@namedef{PY@tok@il}{\def\PY@tc##1{\textcolor[rgb]{0.40,0.40,0.40}{##1}}}
\@namedef{PY@tok@mo}{\def\PY@tc##1{\textcolor[rgb]{0.40,0.40,0.40}{##1}}}
\@namedef{PY@tok@ch}{\let\PY@it=\textit\def\PY@tc##1{\textcolor[rgb]{0.24,0.48,0.48}{##1}}}
\@namedef{PY@tok@cm}{\let\PY@it=\textit\def\PY@tc##1{\textcolor[rgb]{0.24,0.48,0.48}{##1}}}
\@namedef{PY@tok@cpf}{\let\PY@it=\textit\def\PY@tc##1{\textcolor[rgb]{0.24,0.48,0.48}{##1}}}
\@namedef{PY@tok@c1}{\let\PY@it=\textit\def\PY@tc##1{\textcolor[rgb]{0.24,0.48,0.48}{##1}}}
\@namedef{PY@tok@cs}{\let\PY@it=\textit\def\PY@tc##1{\textcolor[rgb]{0.24,0.48,0.48}{##1}}}

\def\PYZbs{\char`\\}
\def\PYZus{\char`\_}
\def\PYZob{\char`\{}
\def\PYZcb{\char`\}}
\def\PYZca{\char`\^}
\def\PYZam{\char`\&}
\def\PYZlt{\char`\<}
\def\PYZgt{\char`\>}
\def\PYZsh{\char`\#}
\def\PYZpc{\char`\%}
\def\PYZdl{\char`\$}
\def\PYZhy{\char`\-}
\def\PYZsq{\char`\'}
\def\PYZdq{\char`\"}
\def\PYZti{\char`\~}
% for compatibility with earlier versions
\def\PYZat{@}
\def\PYZlb{[}
\def\PYZrb{]}
\makeatother


    % For linebreaks inside Verbatim environment from package fancyvrb.
    \makeatletter
        \newbox\Wrappedcontinuationbox
        \newbox\Wrappedvisiblespacebox
        \newcommand*\Wrappedvisiblespace {\textcolor{red}{\textvisiblespace}}
        \newcommand*\Wrappedcontinuationsymbol {\textcolor{red}{\llap{\tiny$\m@th\hookrightarrow$}}}
        \newcommand*\Wrappedcontinuationindent {3ex }
        \newcommand*\Wrappedafterbreak {\kern\Wrappedcontinuationindent\copy\Wrappedcontinuationbox}
        % Take advantage of the already applied Pygments mark-up to insert
        % potential linebreaks for TeX processing.
        %        {, <, #, %, $, ' and ": go to next line.
        %        _, }, ^, &, >, - and ~: stay at end of broken line.
        % Use of \textquotesingle for straight quote.
        \newcommand*\Wrappedbreaksatspecials {%
            \def\PYGZus{\discretionary{\char`\_}{\Wrappedafterbreak}{\char`\_}}%
            \def\PYGZob{\discretionary{}{\Wrappedafterbreak\char`\{}{\char`\{}}%
            \def\PYGZcb{\discretionary{\char`\}}{\Wrappedafterbreak}{\char`\}}}%
            \def\PYGZca{\discretionary{\char`\^}{\Wrappedafterbreak}{\char`\^}}%
            \def\PYGZam{\discretionary{\char`\&}{\Wrappedafterbreak}{\char`\&}}%
            \def\PYGZlt{\discretionary{}{\Wrappedafterbreak\char`\<}{\char`\<}}%
            \def\PYGZgt{\discretionary{\char`\>}{\Wrappedafterbreak}{\char`\>}}%
            \def\PYGZsh{\discretionary{}{\Wrappedafterbreak\char`\#}{\char`\#}}%
            \def\PYGZpc{\discretionary{}{\Wrappedafterbreak\char`\%}{\char`\%}}%
            \def\PYGZdl{\discretionary{}{\Wrappedafterbreak\char`\$}{\char`\$}}%
            \def\PYGZhy{\discretionary{\char`\-}{\Wrappedafterbreak}{\char`\-}}%
            \def\PYGZsq{\discretionary{}{\Wrappedafterbreak\textquotesingle}{\textquotesingle}}%
            \def\PYGZdq{\discretionary{}{\Wrappedafterbreak\char`\"}{\char`\"}}%
            \def\PYGZti{\discretionary{\char`\~}{\Wrappedafterbreak}{\char`\~}}%
        }
        % Some characters . , ; ? ! / are not pygmentized.
        % This macro makes them "active" and they will insert potential linebreaks
        \newcommand*\Wrappedbreaksatpunct {%
            \lccode`\~`\.\lowercase{\def~}{\discretionary{\hbox{\char`\.}}{\Wrappedafterbreak}{\hbox{\char`\.}}}%
            \lccode`\~`\,\lowercase{\def~}{\discretionary{\hbox{\char`\,}}{\Wrappedafterbreak}{\hbox{\char`\,}}}%
            \lccode`\~`\;\lowercase{\def~}{\discretionary{\hbox{\char`\;}}{\Wrappedafterbreak}{\hbox{\char`\;}}}%
            \lccode`\~`\:\lowercase{\def~}{\discretionary{\hbox{\char`\:}}{\Wrappedafterbreak}{\hbox{\char`\:}}}%
            \lccode`\~`\?\lowercase{\def~}{\discretionary{\hbox{\char`\?}}{\Wrappedafterbreak}{\hbox{\char`\?}}}%
            \lccode`\~`\!\lowercase{\def~}{\discretionary{\hbox{\char`\!}}{\Wrappedafterbreak}{\hbox{\char`\!}}}%
            \lccode`\~`\/\lowercase{\def~}{\discretionary{\hbox{\char`\/}}{\Wrappedafterbreak}{\hbox{\char`\/}}}%
            \catcode`\.\active
            \catcode`\,\active
            \catcode`\;\active
            \catcode`\:\active
            \catcode`\?\active
            \catcode`\!\active
            \catcode`\/\active
            \lccode`\~`\~
        }
    \makeatother

    \let\OriginalVerbatim=\Verbatim
    \makeatletter
    \renewcommand{\Verbatim}[1][1]{%
        %\parskip\z@skip
        \sbox\Wrappedcontinuationbox {\Wrappedcontinuationsymbol}%
        \sbox\Wrappedvisiblespacebox {\FV@SetupFont\Wrappedvisiblespace}%
        \def\FancyVerbFormatLine ##1{\hsize\linewidth
            \vtop{\raggedright\hyphenpenalty\z@\exhyphenpenalty\z@
                \doublehyphendemerits\z@\finalhyphendemerits\z@
                \strut ##1\strut}%
        }%
        % If the linebreak is at a space, the latter will be displayed as visible
        % space at end of first line, and a continuation symbol starts next line.
        % Stretch/shrink are however usually zero for typewriter font.
        \def\FV@Space {%
            \nobreak\hskip\z@ plus\fontdimen3\font minus\fontdimen4\font
            \discretionary{\copy\Wrappedvisiblespacebox}{\Wrappedafterbreak}
            {\kern\fontdimen2\font}%
        }%

        % Allow breaks at special characters using \PYG... macros.
        \Wrappedbreaksatspecials
        % Breaks at punctuation characters . , ; ? ! and / need catcode=\active
        \OriginalVerbatim[#1,codes*=\Wrappedbreaksatpunct]%
    }
    \makeatother

    % Exact colors from NB
    \definecolor{incolor}{HTML}{303F9F}
    \definecolor{outcolor}{HTML}{D84315}
    \definecolor{cellborder}{HTML}{CFCFCF}
    \definecolor{cellbackground}{HTML}{F7F7F7}

    % prompt
    \makeatletter
    \newcommand{\boxspacing}{\kern\kvtcb@left@rule\kern\kvtcb@boxsep}
    \makeatother
    \newcommand{\prompt}[4]{
        {\ttfamily\llap{{\color{#2}[#3]:\hspace{3pt}#4}}\vspace{-\baselineskip}}
    }
    

    
    % Prevent overflowing lines due to hard-to-break entities
    \sloppy
    % Setup hyperref package
    \hypersetup{
      breaklinks=true,  % so long urls are correctly broken across lines
      colorlinks=true,
      urlcolor=urlcolor,
      linkcolor=linkcolor,
      citecolor=citecolor,
      }
    % Slightly bigger margins than the latex defaults
    
    \geometry{verbose,tmargin=1in,bmargin=1in,lmargin=1in,rmargin=1in}
    
    

\begin{document}
    
    \maketitle
    
    

    
    \hypertarget{personal-information}{%
\section{Personal Information}\label{personal-information}}

Name: \textbf{Zsofia Katona}

StudentID: \textbf{13160184}

Email:
\href{zsofia.katona@student.uva.nl}{\textbf{zsofia.katona@student.uva.nl}}

Submitted on: \textbf{22.03.2024}

    \hypertarget{data-context}{%
\section{Data Context}\label{data-context}}

\hypertarget{fish-videos-dataset}{%
\subsection{Fish videos dataset}\label{fish-videos-dataset}}

The dataset of fish videos has been recorded and annotated by the
research group. It includes a total of 44 videos, each 10 seconds long,
with a predator attack happening at the middle of the video (in an
approximate 1-second timeframe around the middle). The resolution of the
videos is 1080x1920 pixels, with 120 frames per second (meaning a total
of 1276 frames for each video).

\hypertarget{animal-kingdom}{%
\subsection{Animal Kingdom}\label{animal-kingdom}}

The Animal Kingdom dataset is a publicly available dataset {[}1{]}
containing videos of animals collected from YouTube and annotated for
different video understanding tasks. In this project, the focus is on
the action recognition (AR) task. As described in {[}1{]}, the AR
dataset consists of a total of 50 hours of video clips, with over 850
animal species and 140 fine-grained action classes. As the main focus of
the project is videos of fish predator attacks, the Animal Kingdom AR
dataset will be narrowed down to clips which are related to this domain.
\textbf{The EDA process shown in this notebook identifies and explores
the relevant part of the Animal Kingdom AR dataset.}

{[}1{]} Xun Long Ng, Kian Eng Ong, Qichen Zheng, Yun Ni, Si Yong Yeo,
Jun Liu; Proceedings of the IEEE/CVF Conference on Computer Vision and
Pattern Recognition (CVPR), 2022, pp.~19023-19034

    \hypertarget{data-description}{%
\section{Data Description}\label{data-description}}

Exploring the Animal Kingdom dataset.

    \begin{tcolorbox}[breakable, size=fbox, boxrule=1pt, pad at break*=1mm,colback=cellbackground, colframe=cellborder]
\prompt{In}{incolor}{1}{\boxspacing}
\begin{Verbatim}[commandchars=\\\{\}]
\PY{c+c1}{\PYZsh{} Imports}
\PY{k+kn}{import} \PY{n+nn}{pandas} \PY{k}{as} \PY{n+nn}{pd}
\PY{k+kn}{from} \PY{n+nn}{matplotlib} \PY{k+kn}{import} \PY{n}{pyplot} \PY{k}{as} \PY{n}{plt}
\end{Verbatim}
\end{tcolorbox}

    \hypertarget{importing-metadata-about-action-recognition-ar-dataset}{%
\subsection{Importing metadata about Action Recognition (AR)
dataset}\label{importing-metadata-about-action-recognition-ar-dataset}}

    \begin{tcolorbox}[breakable, size=fbox, boxrule=1pt, pad at break*=1mm,colback=cellbackground, colframe=cellborder]
\prompt{In}{incolor}{2}{\boxspacing}
\begin{Verbatim}[commandchars=\\\{\}]
\PY{c+c1}{\PYZsh{} Importing AR video metadata.}
\PY{n}{video\PYZus{}metadata} \PY{o}{=} \PY{n}{pd}\PY{o}{.}\PY{n}{read\PYZus{}excel}\PY{p}{(}\PY{l+s+s1}{\PYZsq{}}\PY{l+s+s1}{../assets/AK\PYZus{}AR\PYZus{}metadata.xlsx}\PY{l+s+s1}{\PYZsq{}}\PY{p}{,} \PY{n}{sheet\PYZus{}name}\PY{o}{=}\PY{l+s+s1}{\PYZsq{}}\PY{l+s+s1}{AR}\PY{l+s+s1}{\PYZsq{}}\PY{p}{)}

\PY{n}{video\PYZus{}metadata}\PY{o}{.}\PY{n}{head}\PY{p}{(}\PY{p}{)}
\end{Verbatim}
\end{tcolorbox}

            \begin{tcolorbox}[breakable, size=fbox, boxrule=.5pt, pad at break*=1mm, opacityfill=0]
\prompt{Out}{outcolor}{2}{\boxspacing}
\begin{Verbatim}[commandchars=\\\{\}]
   S/N  video\_id   type             labels  \textbackslash{}
0    1  AAACXZTV  train               2,40
1    2  AABCQPTK  train               2,15
2    3  AABKMGHE  train                 78
3    4  AACJGIQR  train  102,1,39,8,120,97
4    5  AACKOHGA  train       2,68,133,116

                                  list\_animal\_action
0  [('Common Crane', 'Attending'), ('Common Crane{\ldots}
1  [('Singing Nightingale', 'Attending'), ('Singi{\ldots}
2                              [('Snake', 'Moving')]
3  [('Mongoose', 'Biting'), ('Black Mamba', 'Sens{\ldots}
4  [('Meerkat', 'Walking'), ('Meerkat', 'Attendin{\ldots}
\end{Verbatim}
\end{tcolorbox}
        
    \begin{tcolorbox}[breakable, size=fbox, boxrule=1pt, pad at break*=1mm,colback=cellbackground, colframe=cellborder]
\prompt{In}{incolor}{8}{\boxspacing}
\begin{Verbatim}[commandchars=\\\{\}]
\PY{c+c1}{\PYZsh{} Ensuring that labels and animal actions are represented as lists.}
\PY{n}{video\PYZus{}metadata}\PY{p}{[}\PY{l+s+s1}{\PYZsq{}}\PY{l+s+s1}{labels}\PY{l+s+s1}{\PYZsq{}}\PY{p}{]} \PY{o}{=} \PY{n}{video\PYZus{}metadata}\PY{p}{[}\PY{l+s+s1}{\PYZsq{}}\PY{l+s+s1}{labels}\PY{l+s+s1}{\PYZsq{}}\PY{p}{]}\PY{o}{.}\PY{n}{apply}\PY{p}{(}\PY{k}{lambda} \PY{n}{labels}\PY{p}{:} \PY{n+nb}{eval}\PY{p}{(}\PY{l+s+s1}{\PYZsq{}}\PY{l+s+s1}{[}\PY{l+s+s1}{\PYZsq{}} \PY{o}{+} \PY{n+nb}{str}\PY{p}{(}\PY{n}{labels}\PY{p}{)} \PY{o}{+} \PY{l+s+s1}{\PYZsq{}}\PY{l+s+s1}{]}\PY{l+s+s1}{\PYZsq{}}\PY{p}{)}\PY{p}{)}

\PY{n}{video\PYZus{}metadata}\PY{p}{[}\PY{l+s+s1}{\PYZsq{}}\PY{l+s+s1}{list\PYZus{}animal\PYZus{}action}\PY{l+s+s1}{\PYZsq{}}\PY{p}{]} \PY{o}{=} \PY{n}{video\PYZus{}metadata}\PY{p}{[}\PY{l+s+s1}{\PYZsq{}}\PY{l+s+s1}{list\PYZus{}animal\PYZus{}action}\PY{l+s+s1}{\PYZsq{}}\PY{p}{]}\PY{o}{.}\PY{n}{apply}\PY{p}{(}\PY{n+nb}{eval}\PY{p}{)}
\end{Verbatim}
\end{tcolorbox}

    \begin{tcolorbox}[breakable, size=fbox, boxrule=1pt, pad at break*=1mm,colback=cellbackground, colframe=cellborder]
\prompt{In}{incolor}{3}{\boxspacing}
\begin{Verbatim}[commandchars=\\\{\}]
\PY{n+nb}{print}\PY{p}{(}\PY{l+s+sa}{f}\PY{l+s+s1}{\PYZsq{}}\PY{l+s+s1}{Number of videos in dataset: }\PY{l+s+si}{\PYZob{}}\PY{n+nb}{len}\PY{p}{(}\PY{n}{video\PYZus{}metadata}\PY{p}{)}\PY{l+s+si}{\PYZcb{}}\PY{l+s+s1}{\PYZsq{}}\PY{p}{)}

\PY{n}{distinct\PYZus{}IDs} \PY{o}{=} \PY{n}{video\PYZus{}metadata}\PY{p}{[}\PY{l+s+s1}{\PYZsq{}}\PY{l+s+s1}{video\PYZus{}id}\PY{l+s+s1}{\PYZsq{}}\PY{p}{]}\PY{o}{.}\PY{n}{nunique}\PY{p}{(}\PY{p}{)}

\PY{n+nb}{print}\PY{p}{(}\PY{l+s+sa}{f}\PY{l+s+s1}{\PYZsq{}}\PY{l+s+s1}{Number of distinct video IDs: }\PY{l+s+si}{\PYZob{}}\PY{n}{distinct\PYZus{}IDs}\PY{l+s+si}{\PYZcb{}}\PY{l+s+s1}{\PYZsq{}}\PY{p}{)}
\end{Verbatim}
\end{tcolorbox}

    \begin{Verbatim}[commandchars=\\\{\}]
Number of videos in dataset: 30100
Number of distinct video IDs: 30100
    \end{Verbatim}

    As can be seen, actions in the videos are represented in a list of
(ACTOR, ACTION) tuples.

    \begin{tcolorbox}[breakable, size=fbox, boxrule=1pt, pad at break*=1mm,colback=cellbackground, colframe=cellborder]
\prompt{In}{incolor}{4}{\boxspacing}
\begin{Verbatim}[commandchars=\\\{\}]
\PY{c+c1}{\PYZsh{} Importing metadata about animal species in the dataset.}
\PY{n}{animal\PYZus{}data} \PY{o}{=} \PY{n}{pd}\PY{o}{.}\PY{n}{read\PYZus{}excel}\PY{p}{(}\PY{l+s+s1}{\PYZsq{}}\PY{l+s+s1}{../assets/AK\PYZus{}AR\PYZus{}metadata.xlsx}\PY{l+s+s1}{\PYZsq{}}\PY{p}{,} \PY{n}{sheet\PYZus{}name}\PY{o}{=}\PY{l+s+s1}{\PYZsq{}}\PY{l+s+s1}{Animal}\PY{l+s+s1}{\PYZsq{}}\PY{p}{)}

\PY{n}{animal\PYZus{}data}\PY{o}{.}\PY{n}{head}\PY{p}{(}\PY{p}{)}
\end{Verbatim}
\end{tcolorbox}

            \begin{tcolorbox}[breakable, size=fbox, boxrule=.5pt, pad at break*=1mm, opacityfill=0]
\prompt{Out}{outcolor}{4}{\boxspacing}
\begin{Verbatim}[commandchars=\\\{\}]
                         Animal Parent Class                   Class  \textbackslash{}
0                       Abalone   Sea animal  Sea animal / Shellfish
1     Acyrthosiphon pisum Aphid       Insect                  Insect
2        Aedes aegypti Mosquito       Insect                  Insect
3  Aedes aegypti Mosquito Larva       Insect   Insect / Larva / Pupa
4             Aesculapian Snake      Reptile         Reptile / Snake

                                       Sub-Class Young Generic name
0  Shellfish / Abalone / Clam / Mussel / Scallop   NaN          NaN
1                          Ant / Termite / Aphid   NaN          NaN
2                           Mosquito / Crane fly   NaN          NaN
3                                   Larva / Pupa   Yes          NaN
4                 Snake / Cobra / Viper / Python   NaN          NaN
\end{Verbatim}
\end{tcolorbox}
        
    \hypertarget{finding-relevant-videos}{%
\section{Finding relevant videos}\label{finding-relevant-videos}}

    \hypertarget{finding-videos-with-relevant-species}{%
\subsection{Finding videos with relevant
species}\label{finding-videos-with-relevant-species}}

Identifying videos showing fish.

    \begin{tcolorbox}[breakable, size=fbox, boxrule=1pt, pad at break*=1mm,colback=cellbackground, colframe=cellborder]
\prompt{In}{incolor}{5}{\boxspacing}
\begin{Verbatim}[commandchars=\\\{\}]
\PY{c+c1}{\PYZsh{} Saving list of fish species in the dataset.}
\PY{n}{fish} \PY{o}{=} \PY{n+nb}{list}\PY{p}{(}\PY{n}{animal\PYZus{}data}\PY{p}{[}\PY{n}{animal\PYZus{}data}\PY{p}{[}\PY{l+s+s1}{\PYZsq{}}\PY{l+s+s1}{Class}\PY{l+s+s1}{\PYZsq{}}\PY{p}{]} \PY{o}{==} \PY{l+s+s1}{\PYZsq{}}\PY{l+s+s1}{Fish}\PY{l+s+s1}{\PYZsq{}}\PY{p}{]}\PY{p}{[}\PY{l+s+s1}{\PYZsq{}}\PY{l+s+s1}{Animal}\PY{l+s+s1}{\PYZsq{}}\PY{p}{]}\PY{p}{)}
\end{Verbatim}
\end{tcolorbox}

    \begin{tcolorbox}[breakable, size=fbox, boxrule=1pt, pad at break*=1mm,colback=cellbackground, colframe=cellborder]
\prompt{In}{incolor}{6}{\boxspacing}
\begin{Verbatim}[commandchars=\\\{\}]
\PY{n+nb}{print}\PY{p}{(}\PY{l+s+s1}{\PYZsq{}}\PY{l+s+s1}{Number of fish species in dataset:}\PY{l+s+s1}{\PYZsq{}}\PY{p}{,} \PY{n+nb}{len}\PY{p}{(}\PY{n}{fish}\PY{p}{)}\PY{p}{)}
\PY{n+nb}{print}\PY{p}{(}\PY{l+s+s1}{\PYZsq{}}\PY{l+s+s1}{Examples:}\PY{l+s+s1}{\PYZsq{}}\PY{p}{,} \PY{n}{fish}\PY{p}{[}\PY{p}{:}\PY{l+m+mi}{3}\PY{p}{]}\PY{p}{)}
\end{Verbatim}
\end{tcolorbox}

    \begin{Verbatim}[commandchars=\\\{\}]
Number of fish species in dataset: 59
Examples: ['Archer Fish', 'Atlantic Blue Tang Fish', 'Barracuda Fish']
    \end{Verbatim}

    \begin{tcolorbox}[breakable, size=fbox, boxrule=1pt, pad at break*=1mm,colback=cellbackground, colframe=cellborder]
\prompt{In}{incolor}{18}{\boxspacing}
\begin{Verbatim}[commandchars=\\\{\}]
\PY{k}{def} \PY{n+nf}{check\PYZus{}species\PYZus{}from\PYZus{}list}\PY{p}{(}\PY{n}{actions}\PY{p}{,} \PY{n}{list\PYZus{}of\PYZus{}species}\PY{p}{)}\PY{p}{:}
\PY{+w}{    }\PY{l+s+sd}{\PYZsq{}\PYZsq{}\PYZsq{}Checks if a video contains action performed by an animal from a list of species.\PYZsq{}\PYZsq{}\PYZsq{}}

    \PY{k}{for} \PY{n}{action\PYZus{}tuple} \PY{o+ow}{in} \PY{n}{actions}\PY{p}{:}
        \PY{n}{species} \PY{o}{=} \PY{n}{action\PYZus{}tuple}\PY{p}{[}\PY{l+m+mi}{0}\PY{p}{]}

        \PY{k}{if} \PY{n}{species} \PY{o+ow}{in} \PY{n}{list\PYZus{}of\PYZus{}species}\PY{p}{:}
            \PY{k}{return} \PY{k+kc}{True}
        
    \PY{k}{return} \PY{k+kc}{False}
\end{Verbatim}
\end{tcolorbox}

    \begin{tcolorbox}[breakable, size=fbox, boxrule=1pt, pad at break*=1mm,colback=cellbackground, colframe=cellborder]
\prompt{In}{incolor}{19}{\boxspacing}
\begin{Verbatim}[commandchars=\\\{\}]
\PY{n}{fish\PYZus{}videos} \PY{o}{=} \PY{n}{video\PYZus{}metadata}\PY{p}{[}\PY{n}{video\PYZus{}metadata}\PY{p}{[}\PY{l+s+s1}{\PYZsq{}}\PY{l+s+s1}{list\PYZus{}animal\PYZus{}action}\PY{l+s+s1}{\PYZsq{}}\PY{p}{]}\PY{o}{.}\PY{n}{apply}\PY{p}{(}\PY{k}{lambda} \PY{n}{actions}\PY{p}{:} \PY{n}{check\PYZus{}species\PYZus{}from\PYZus{}list}\PY{p}{(}\PY{n}{actions}\PY{p}{,} \PY{n}{fish}\PY{p}{)}\PY{p}{)}\PY{p}{]}
\end{Verbatim}
\end{tcolorbox}

    \begin{tcolorbox}[breakable, size=fbox, boxrule=1pt, pad at break*=1mm,colback=cellbackground, colframe=cellborder]
\prompt{In}{incolor}{20}{\boxspacing}
\begin{Verbatim}[commandchars=\\\{\}]
\PY{n+nb}{print}\PY{p}{(}\PY{l+s+s1}{\PYZsq{}}\PY{l+s+s1}{Number of fish videos in dataset:}\PY{l+s+s1}{\PYZsq{}}\PY{p}{,} \PY{n+nb}{len}\PY{p}{(}\PY{n}{fish\PYZus{}videos}\PY{p}{)}\PY{p}{)}
\end{Verbatim}
\end{tcolorbox}

    \begin{Verbatim}[commandchars=\\\{\}]
Number of fish videos in dataset: 887
    \end{Verbatim}

    \hypertarget{finding-videos-with-relevant-actions}{%
\subsection{Finding videos with relevant
actions}\label{finding-videos-with-relevant-actions}}

In our case, these are actions related to predation, namely:

\begin{itemize}
\tightlist
\item
  1: \textbf{Attacking}
\item
  7: \textbf{Being eaten}
\item
  14: \textbf{Chasing}
\item
  47: \textbf{Fighting}
\item
  8: \textbf{Biting}
\item
  51: \textbf{Fleeing}
\item
  96: \textbf{Retaliating}
\item
  120: \textbf{Struggling}
\end{itemize}

    \begin{tcolorbox}[breakable, size=fbox, boxrule=1pt, pad at break*=1mm,colback=cellbackground, colframe=cellborder]
\prompt{In}{incolor}{21}{\boxspacing}
\begin{Verbatim}[commandchars=\\\{\}]
\PY{n}{relevant\PYZus{}actions} \PY{o}{=} \PY{p}{[}\PY{l+s+s1}{\PYZsq{}}\PY{l+s+s1}{Attacking}\PY{l+s+s1}{\PYZsq{}}\PY{p}{,} \PY{l+s+s1}{\PYZsq{}}\PY{l+s+s1}{Chasing}\PY{l+s+s1}{\PYZsq{}}\PY{p}{,} \PY{l+s+s1}{\PYZsq{}}\PY{l+s+s1}{Fighting}\PY{l+s+s1}{\PYZsq{}}\PY{p}{,} \PY{l+s+s1}{\PYZsq{}}\PY{l+s+s1}{Biting}\PY{l+s+s1}{\PYZsq{}}\PY{p}{,} \PYZbs{}
                    \PY{l+s+s1}{\PYZsq{}}\PY{l+s+s1}{Being eaten}\PY{l+s+s1}{\PYZsq{}}\PY{p}{,} \PY{l+s+s1}{\PYZsq{}}\PY{l+s+s1}{Fleeing}\PY{l+s+s1}{\PYZsq{}}\PY{p}{,} \PY{l+s+s1}{\PYZsq{}}\PY{l+s+s1}{Retaliating}\PY{l+s+s1}{\PYZsq{}}\PY{p}{,} \PY{l+s+s1}{\PYZsq{}}\PY{l+s+s1}{Struggling}\PY{l+s+s1}{\PYZsq{}}\PY{p}{]}
\end{Verbatim}
\end{tcolorbox}

    \begin{tcolorbox}[breakable, size=fbox, boxrule=1pt, pad at break*=1mm,colback=cellbackground, colframe=cellborder]
\prompt{In}{incolor}{23}{\boxspacing}
\begin{Verbatim}[commandchars=\\\{\}]
\PY{n}{action\PYZus{}data} \PY{o}{=} \PY{n}{pd}\PY{o}{.}\PY{n}{read\PYZus{}excel}\PY{p}{(}\PY{l+s+s1}{\PYZsq{}}\PY{l+s+s1}{../assets/AK\PYZus{}AR\PYZus{}metadata.xlsx}\PY{l+s+s1}{\PYZsq{}}\PY{p}{,} \PY{n}{sheet\PYZus{}name}\PY{o}{=}\PY{l+s+s1}{\PYZsq{}}\PY{l+s+s1}{Action}\PY{l+s+s1}{\PYZsq{}}\PY{p}{)}

\PY{k}{for} \PY{n}{i} \PY{o+ow}{in} \PY{n}{action\PYZus{}data}\PY{p}{[}\PY{n}{action\PYZus{}data}\PY{p}{[}\PY{l+s+s1}{\PYZsq{}}\PY{l+s+s1}{Action}\PY{l+s+s1}{\PYZsq{}}\PY{p}{]}\PY{o}{.}\PY{n}{isin}\PY{p}{(}\PY{n}{relevant\PYZus{}actions}\PY{p}{)}\PY{p}{]}\PY{o}{.}\PY{n}{index}\PY{p}{:}
    \PY{n+nb}{print}\PY{p}{(}\PY{l+s+s1}{\PYZsq{}}\PY{l+s+s1}{\PYZhy{}\PYZhy{}}\PY{l+s+s1}{\PYZsq{}} \PY{o}{+} \PY{n}{action\PYZus{}data}\PY{p}{[}\PY{l+s+s1}{\PYZsq{}}\PY{l+s+s1}{Action}\PY{l+s+s1}{\PYZsq{}}\PY{p}{]}\PY{p}{[}\PY{n}{i}\PY{p}{]}\PY{p}{)}
    \PY{n+nb}{print}\PY{p}{(}\PY{n}{action\PYZus{}data}\PY{p}{[}\PY{l+s+s1}{\PYZsq{}}\PY{l+s+s1}{Description}\PY{l+s+s1}{\PYZsq{}}\PY{p}{]}\PY{p}{[}\PY{n}{i}\PY{p}{]}\PY{p}{)}
    \PY{n+nb}{print}\PY{p}{(}\PY{p}{)}
\end{Verbatim}
\end{tcolorbox}

    \begin{Verbatim}[commandchars=\\\{\}]
--Attacking
Animal rushes and leaps on another another with bites, pecks or kicks

--Chasing
Animal "runs" after the fleeing animal, however no biting occurs

--Fighting
Include wrestling

--Fleeing
Different from escaping, animal 'runs' away from its predator or danger, often
with rapid change of direction but without having been caught first

--Retaliating
Animal makes a defending attack (i.e., "attacks" its attacker back). Include
ramming by goats

--Struggling
Animal struggles from the clutch, grip, or bite by its predator

--Biting
Animal sinks its teeth into the object or animal, but does not eat / feed / chew
on the object or food

--Being eaten
nan

    \end{Verbatim}

    \hypertarget{collecting-relevant-videos}{%
\subsection{Collecting relevant
videos}\label{collecting-relevant-videos}}

Please note that during the selection of relevant videos, only clips in
which fish are the performers of attack-related actions (either on the
predator or prey side) are selected. However, the clip might contain
animals other than fish, and they might be involved in the attack as
well (e.g., snake eating fish).

    \begin{tcolorbox}[breakable, size=fbox, boxrule=1pt, pad at break*=1mm,colback=cellbackground, colframe=cellborder]
\prompt{In}{incolor}{25}{\boxspacing}
\begin{Verbatim}[commandchars=\\\{\}]
\PY{k}{def} \PY{n+nf}{is\PYZus{}relevant\PYZus{}video}\PY{p}{(}\PY{n}{actions\PYZus{}in\PYZus{}video}\PY{p}{,} \PY{n}{relevant\PYZus{}actions}\PY{p}{,} \PY{n}{relevant\PYZus{}species}\PY{p}{)}\PY{p}{:}
\PY{+w}{    }\PY{l+s+sd}{\PYZsq{}\PYZsq{}\PYZsq{}Checks if a video contains a relevant action performed by a releveant species.\PYZsq{}\PYZsq{}\PYZsq{}}

    \PY{k}{for} \PY{n}{action\PYZus{}tuple} \PY{o+ow}{in} \PY{n}{actions\PYZus{}in\PYZus{}video}\PY{p}{:}
        \PY{n}{species}\PY{p}{,} \PY{n}{action} \PY{o}{=} \PY{n}{action\PYZus{}tuple}

        \PY{k}{if} \PY{n}{species} \PY{o+ow}{in} \PY{n}{relevant\PYZus{}species} \PY{o+ow}{and} \PY{n}{action} \PY{o+ow}{in} \PY{n}{relevant\PYZus{}actions}\PY{p}{:}
            \PY{k}{return} \PY{k+kc}{True}
        
    \PY{k}{return} \PY{k+kc}{False}
\end{Verbatim}
\end{tcolorbox}

    \begin{tcolorbox}[breakable, size=fbox, boxrule=1pt, pad at break*=1mm,colback=cellbackground, colframe=cellborder]
\prompt{In}{incolor}{26}{\boxspacing}
\begin{Verbatim}[commandchars=\\\{\}]
\PY{c+c1}{\PYZsh{} Narrowing the dataset to only relevant fish videos (i.e., the ones containing attack\PYZhy{}related action performed by fish.)}
\PY{n}{relevant\PYZus{}videos} \PY{o}{=} \PY{n}{fish\PYZus{}videos}\PY{p}{[}\PY{n}{fish\PYZus{}videos}\PY{p}{[}\PY{l+s+s1}{\PYZsq{}}\PY{l+s+s1}{list\PYZus{}animal\PYZus{}action}\PY{l+s+s1}{\PYZsq{}}\PY{p}{]}\PY{o}{.}\PY{n}{apply}\PY{p}{(}\PY{k}{lambda} \PY{n}{actions}\PY{p}{:} \PYZbs{}
                                                                      \PY{n}{is\PYZus{}relevant\PYZus{}video}\PY{p}{(}\PY{n}{actions}\PY{p}{,} \PY{n}{relevant\PYZus{}actions}\PY{p}{,} \PY{n}{fish}\PY{p}{)}\PY{p}{)}\PY{p}{]}

\PY{n+nb}{print}\PY{p}{(}\PY{l+s+s1}{\PYZsq{}}\PY{l+s+s1}{Number of relevant videos in dataset:}\PY{l+s+s1}{\PYZsq{}}\PY{p}{,} \PY{n+nb}{len}\PY{p}{(}\PY{n}{relevant\PYZus{}videos}\PY{p}{)}\PY{p}{)}
\end{Verbatim}
\end{tcolorbox}

    \begin{Verbatim}[commandchars=\\\{\}]
Number of relevant videos in dataset: 81
    \end{Verbatim}

    \begin{tcolorbox}[breakable, size=fbox, boxrule=1pt, pad at break*=1mm,colback=cellbackground, colframe=cellborder]
\prompt{In}{incolor}{30}{\boxspacing}
\begin{Verbatim}[commandchars=\\\{\}]
\PY{n+nb}{print}\PY{p}{(}\PY{l+s+s1}{\PYZsq{}}\PY{l+s+s1}{Number of relevant training samples:}\PY{l+s+s1}{\PYZsq{}}\PY{p}{,} \PY{n+nb}{len}\PY{p}{(}\PY{n}{relevant\PYZus{}videos}\PY{p}{[}\PY{n}{relevant\PYZus{}videos}\PY{p}{[}\PY{l+s+s1}{\PYZsq{}}\PY{l+s+s1}{type}\PY{l+s+s1}{\PYZsq{}}\PY{p}{]} \PY{o}{==} \PY{l+s+s1}{\PYZsq{}}\PY{l+s+s1}{train}\PY{l+s+s1}{\PYZsq{}}\PY{p}{]}\PY{p}{)}\PY{p}{)}
\PY{n+nb}{print}\PY{p}{(}\PY{l+s+s1}{\PYZsq{}}\PY{l+s+s1}{Number of relevant test samples:}\PY{l+s+s1}{\PYZsq{}}\PY{p}{,} \PY{n+nb}{len}\PY{p}{(}\PY{n}{relevant\PYZus{}videos}\PY{p}{[}\PY{n}{relevant\PYZus{}videos}\PY{p}{[}\PY{l+s+s1}{\PYZsq{}}\PY{l+s+s1}{type}\PY{l+s+s1}{\PYZsq{}}\PY{p}{]} \PY{o}{==} \PY{l+s+s1}{\PYZsq{}}\PY{l+s+s1}{test}\PY{l+s+s1}{\PYZsq{}}\PY{p}{]}\PY{p}{)}\PY{p}{)}
\end{Verbatim}
\end{tcolorbox}

    \begin{Verbatim}[commandchars=\\\{\}]
Number of relevant training samples: 74
Number of relevant test samples: 7
    \end{Verbatim}

    \hypertarget{inspecting-relevant-action-videos}{%
\section{Inspecting relevant action
videos}\label{inspecting-relevant-action-videos}}

    \begin{tcolorbox}[breakable, size=fbox, boxrule=1pt, pad at break*=1mm,colback=cellbackground, colframe=cellborder]
\prompt{In}{incolor}{34}{\boxspacing}
\begin{Verbatim}[commandchars=\\\{\}]
\PY{c+c1}{\PYZsh{} Setting up counters to record the distribution of actions and species in the relevant videos.}
\PY{n}{action\PYZus{}counter} \PY{o}{=} \PY{p}{\PYZob{}}\PY{n}{action} \PY{p}{:} \PY{l+m+mi}{0} \PY{k}{for} \PY{n}{action} \PY{o+ow}{in} \PY{n}{relevant\PYZus{}actions}\PY{p}{\PYZcb{}}

\PY{n}{species\PYZus{}counter} \PY{o}{=} \PY{p}{\PYZob{}}\PY{p}{\PYZcb{}}
\end{Verbatim}
\end{tcolorbox}

    \begin{tcolorbox}[breakable, size=fbox, boxrule=1pt, pad at break*=1mm,colback=cellbackground, colframe=cellborder]
\prompt{In}{incolor}{35}{\boxspacing}
\begin{Verbatim}[commandchars=\\\{\}]
\PY{k}{for} \PY{n}{i} \PY{o+ow}{in} \PY{n}{relevant\PYZus{}videos}\PY{o}{.}\PY{n}{index}\PY{p}{:}
    \PY{n}{actions} \PY{o}{=} \PY{n}{relevant\PYZus{}videos}\PY{o}{.}\PY{n}{loc}\PY{p}{[}\PY{n}{i}\PY{p}{,} \PY{l+s+s1}{\PYZsq{}}\PY{l+s+s1}{list\PYZus{}animal\PYZus{}action}\PY{l+s+s1}{\PYZsq{}}\PY{p}{]}

    \PY{k}{for} \PY{n}{action\PYZus{}tuple} \PY{o+ow}{in} \PY{n}{actions}\PY{p}{:}
        \PY{n}{species}\PY{p}{,} \PY{n}{action} \PY{o}{=} \PY{n}{action\PYZus{}tuple}

        \PY{k}{if} \PY{n}{species} \PY{o+ow}{in} \PY{n}{fish} \PY{o+ow}{and} \PY{n}{action} \PY{o+ow}{in} \PY{n}{relevant\PYZus{}actions}\PY{p}{:} 
            \PY{k}{if} \PY{n}{species} \PY{o+ow}{not} \PY{o+ow}{in} \PY{n}{species\PYZus{}counter}\PY{p}{:}
                \PY{n}{species\PYZus{}counter}\PY{p}{[}\PY{n}{species}\PY{p}{]} \PY{o}{=} \PY{l+m+mi}{1}

            \PY{k}{else}\PY{p}{:}
                \PY{n}{species\PYZus{}counter}\PY{p}{[}\PY{n}{species}\PY{p}{]} \PY{o}{+}\PY{o}{=} \PY{l+m+mi}{1}

            \PY{n}{action\PYZus{}counter}\PY{p}{[}\PY{n}{action}\PY{p}{]} \PY{o}{+}\PY{o}{=} \PY{l+m+mi}{1}
\end{Verbatim}
\end{tcolorbox}

    \hypertarget{distribution-of-relevant-action-types}{%
\subsection{Distribution of relevant action
types}\label{distribution-of-relevant-action-types}}

    \begin{tcolorbox}[breakable, size=fbox, boxrule=1pt, pad at break*=1mm,colback=cellbackground, colframe=cellborder]
\prompt{In}{incolor}{51}{\boxspacing}
\begin{Verbatim}[commandchars=\\\{\}]
\PY{n}{sorted\PYZus{}items} \PY{o}{=} \PY{n+nb}{sorted}\PY{p}{(}\PY{n}{action\PYZus{}counter}\PY{o}{.}\PY{n}{items}\PY{p}{(}\PY{p}{)}\PY{p}{,} \PY{n}{key}\PY{o}{=}\PY{k}{lambda} \PY{n}{x}\PY{p}{:} \PY{n}{x}\PY{p}{[}\PY{l+m+mi}{1}\PY{p}{]}\PY{p}{,} \PY{n}{reverse}\PY{o}{=}\PY{k+kc}{True}\PY{p}{)}

\PY{n}{actions}\PY{p}{,} \PY{n}{occurrences} \PY{o}{=} \PY{n+nb}{zip}\PY{p}{(}\PY{o}{*}\PY{n}{sorted\PYZus{}items}\PY{p}{)}

\PY{n}{plt}\PY{o}{.}\PY{n}{bar}\PY{p}{(}\PY{n}{actions}\PY{p}{,} \PY{n}{occurrences}\PY{p}{,} \PY{n}{color}\PY{o}{=}\PY{l+s+s1}{\PYZsq{}}\PY{l+s+s1}{cornflowerblue}\PY{l+s+s1}{\PYZsq{}}\PY{p}{)}
\PY{n}{plt}\PY{o}{.}\PY{n}{xlabel}\PY{p}{(}\PY{l+s+s1}{\PYZsq{}}\PY{l+s+s1}{Actions}\PY{l+s+s1}{\PYZsq{}}\PY{p}{)}
\PY{n}{plt}\PY{o}{.}\PY{n}{ylabel}\PY{p}{(}\PY{l+s+s1}{\PYZsq{}}\PY{l+s+s1}{Occurrences}\PY{l+s+s1}{\PYZsq{}}\PY{p}{)}
\PY{n}{plt}\PY{o}{.}\PY{n}{title}\PY{p}{(}\PY{l+s+s1}{\PYZsq{}}\PY{l+s+s1}{Distribution of Action Occurrences}\PY{l+s+s1}{\PYZsq{}}\PY{p}{)}
\PY{n}{plt}\PY{o}{.}\PY{n}{xticks}\PY{p}{(}\PY{n}{rotation}\PY{o}{=}\PY{l+m+mi}{45}\PY{p}{)}
\PY{n}{plt}\PY{o}{.}\PY{n}{show}\PY{p}{(}\PY{p}{)}
\end{Verbatim}
\end{tcolorbox}

    \begin{center}
    \adjustimage{max size={0.9\linewidth}{0.9\paperheight}}{ZKatona_thesis_EDA_files/ZKatona_thesis_EDA_28_0.png}
    \end{center}
    { \hspace*{\fill} \\}
    
    \hypertarget{distribution-of-different-fish-species-in-relevant-actions}{%
\subsection{Distribution of different fish species in relevant
actions}\label{distribution-of-different-fish-species-in-relevant-actions}}

    \begin{tcolorbox}[breakable, size=fbox, boxrule=1pt, pad at break*=1mm,colback=cellbackground, colframe=cellborder]
\prompt{In}{incolor}{38}{\boxspacing}
\begin{Verbatim}[commandchars=\\\{\}]
\PY{n}{sorted\PYZus{}items} \PY{o}{=} \PY{n+nb}{sorted}\PY{p}{(}\PY{n}{species\PYZus{}counter}\PY{o}{.}\PY{n}{items}\PY{p}{(}\PY{p}{)}\PY{p}{,} \PY{n}{key}\PY{o}{=}\PY{k}{lambda} \PY{n}{x}\PY{p}{:} \PY{n}{x}\PY{p}{[}\PY{l+m+mi}{1}\PY{p}{]}\PY{p}{,} \PY{n}{reverse}\PY{o}{=}\PY{k+kc}{False}\PY{p}{)}

\PY{n}{species}\PY{p}{,} \PY{n}{occurrences} \PY{o}{=} \PY{n+nb}{zip}\PY{p}{(}\PY{o}{*}\PY{n}{sorted\PYZus{}items}\PY{p}{)}

\PY{n}{plt}\PY{o}{.}\PY{n}{barh}\PY{p}{(}\PY{n}{species}\PY{p}{,} \PYZbs{}
        \PY{n}{occurrences}\PY{p}{,} \PYZbs{}
            \PY{n}{color}\PY{o}{=}\PY{l+s+s1}{\PYZsq{}}\PY{l+s+s1}{cornflowerblue}\PY{l+s+s1}{\PYZsq{}}\PY{p}{)}
\PY{n}{plt}\PY{o}{.}\PY{n}{ylabel}\PY{p}{(}\PY{l+s+s1}{\PYZsq{}}\PY{l+s+s1}{Species}\PY{l+s+s1}{\PYZsq{}}\PY{p}{)}
\PY{n}{plt}\PY{o}{.}\PY{n}{xlabel}\PY{p}{(}\PY{l+s+s1}{\PYZsq{}}\PY{l+s+s1}{Action occurrences}\PY{l+s+s1}{\PYZsq{}}\PY{p}{)}
\PY{n}{plt}\PY{o}{.}\PY{n}{title}\PY{p}{(}\PY{l+s+s1}{\PYZsq{}}\PY{l+s+s1}{Distribution of Species Occurences}\PY{l+s+s1}{\PYZsq{}}\PY{p}{)}
\PY{n}{plt}\PY{o}{.}\PY{n}{show}\PY{p}{(}\PY{p}{)}
\end{Verbatim}
\end{tcolorbox}

    \begin{center}
    \adjustimage{max size={0.9\linewidth}{0.9\paperheight}}{ZKatona_thesis_EDA_files/ZKatona_thesis_EDA_30_0.png}
    \end{center}
    { \hspace*{\fill} \\}
    
    \hypertarget{exploring-possibility-of-baseline-actions.}{%
\subsection{Exploring possibility of ``baseline''
actions.}\label{exploring-possibility-of-baseline-actions.}}

The project builds on a binary classification task: predicting whether a
video clip contains a predator attack or not. While the positive
(``attack'') examples can be collected using the action categories shown
above, negative samples can either contain any fish video, or any fish
video which contains a fish labelled as swimming. The latter is an
option to ensure that only videos of moving fish are used. This section
aims to explore the prevalence of ``swimming'' labels in fish videos of
the AR dataset.

    \begin{tcolorbox}[breakable, size=fbox, boxrule=1pt, pad at break*=1mm,colback=cellbackground, colframe=cellborder]
\prompt{In}{incolor}{49}{\boxspacing}
\begin{Verbatim}[commandchars=\\\{\}]
\PY{c+c1}{\PYZsh{} Counting videos in which fish swim:}

\PY{n}{relevant\PYZus{}video\PYZus{}ids} \PY{o}{=} \PY{n+nb}{list}\PY{p}{(}\PY{n}{relevant\PYZus{}videos}\PY{p}{[}\PY{l+s+s1}{\PYZsq{}}\PY{l+s+s1}{video\PYZus{}id}\PY{l+s+s1}{\PYZsq{}}\PY{p}{]}\PY{p}{)}

\PY{n}{swim\PYZus{}counter} \PY{o}{=} \PY{l+m+mi}{0}
\PY{n}{swim\PYZus{}counter\PYZus{}in\PYZus{}predation} \PY{o}{=} \PY{l+m+mi}{0} \PY{c+c1}{\PYZsh{} Number of videos in which there is fish predation AND swimming.}

\PY{k}{for} \PY{n}{i} \PY{o+ow}{in} \PY{n}{fish\PYZus{}videos}\PY{o}{.}\PY{n}{index}\PY{p}{:}
    \PY{n}{actions} \PY{o}{=} \PY{n}{fish\PYZus{}videos}\PY{o}{.}\PY{n}{loc}\PY{p}{[}\PY{n}{i}\PY{p}{,} \PY{l+s+s1}{\PYZsq{}}\PY{l+s+s1}{list\PYZus{}animal\PYZus{}action}\PY{l+s+s1}{\PYZsq{}}\PY{p}{]}

    \PY{k}{for} \PY{n}{action\PYZus{}tuple} \PY{o+ow}{in} \PY{n}{actions}\PY{p}{:}
        \PY{n}{species}\PY{p}{,} \PY{n}{action} \PY{o}{=} \PY{n}{action\PYZus{}tuple}

        \PY{k}{if} \PY{n}{species} \PY{o+ow}{in} \PY{n}{fish} \PY{o+ow}{and} \PY{n}{action} \PY{o}{==} \PY{l+s+s1}{\PYZsq{}}\PY{l+s+s1}{Swimming}\PY{l+s+s1}{\PYZsq{}}\PY{p}{:}
            \PY{n}{swim\PYZus{}counter} \PY{o}{+}\PY{o}{=} \PY{l+m+mi}{1}

            \PY{k}{if} \PY{n}{fish\PYZus{}videos}\PY{o}{.}\PY{n}{loc}\PY{p}{[}\PY{n}{i}\PY{p}{,} \PY{l+s+s1}{\PYZsq{}}\PY{l+s+s1}{video\PYZus{}id}\PY{l+s+s1}{\PYZsq{}}\PY{p}{]} \PY{o+ow}{in} \PY{n}{relevant\PYZus{}video\PYZus{}ids}\PY{p}{:}
                \PY{n}{swim\PYZus{}counter\PYZus{}in\PYZus{}predation} \PY{o}{+}\PY{o}{=} \PY{l+m+mi}{1}

            \PY{k}{continue} \PY{c+c1}{\PYZsh{} Count each video only once.}
\end{Verbatim}
\end{tcolorbox}

    \begin{tcolorbox}[breakable, size=fbox, boxrule=1pt, pad at break*=1mm,colback=cellbackground, colframe=cellborder]
\prompt{In}{incolor}{50}{\boxspacing}
\begin{Verbatim}[commandchars=\\\{\}]
\PY{n+nb}{print}\PY{p}{(}\PY{l+s+sa}{f}\PY{l+s+s1}{\PYZsq{}}\PY{l+s+s1}{Number of videos with fish labelled as swimming: }\PY{l+s+si}{\PYZob{}}\PY{n}{swim\PYZus{}counter}\PY{l+s+si}{\PYZcb{}}\PY{l+s+s1}{\PYZsq{}}\PY{p}{)}
\PY{n+nb}{print}\PY{p}{(}\PY{l+s+sa}{f}\PY{l+s+s1}{\PYZsq{}}\PY{l+s+s1}{Number of videos with fish in predation and fish labelled as swimming: }\PY{l+s+si}{\PYZob{}}\PY{n}{swim\PYZus{}counter\PYZus{}in\PYZus{}predation}\PY{l+s+si}{\PYZcb{}}\PY{l+s+s1}{\PYZsq{}}\PY{p}{)}
\end{Verbatim}
\end{tcolorbox}

    \begin{Verbatim}[commandchars=\\\{\}]
Number of videos with fish labelled as swimming: 504
Number of videos with fish in predation and fish labelled as swimming: 38
    \end{Verbatim}


    % Add a bibliography block to the postdoc
    
    
    
\end{document}
